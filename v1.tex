\documentclass[10pt]{article}
\usepackage{amsmath, amssymb, graphicx}
\usepackage{booktabs}
\usepackage{listings}
\usepackage{geometry}
\usepackage[encapsulated]{CJK}
\usepackage{enumitem}
\usepackage{tikz}
\usetikzlibrary{arrows.meta, positioning}
\newcommand*\circled[1]{%
  \tikz[baseline=(char.base)]{
    \node[shape=circle,draw,inner sep=2pt] (char) {#1};
  }%
}
\usepackage{color}
\lstset{ %
  language=C++,                
  basicstyle=\footnotesize,       
  numbers=left,                   
  numberstyle=\footnotesize,      
  stepnumber=1,                   
  numbersep=5pt,                  
  backgroundcolor=\color{white},  
  showspaces=false,               
  showstringspaces=false,         
  showtabs=false,                 
  frame=single,           
  tabsize=2,          
  captionpos=b,           
  breaklines=true,        
  breakatwhitespace=false,    
  escapeinside={\%*}{*)}          
}

% --- 統一所有 enumerate 的編號樣式 ---
\setlist[enumerate]{%
  label=(\arabic*),   % 例:(1) (2) (3)
  ref=\arabic*,       % 交叉引用時顯示 1 2 3
  leftmargin=1.5em,   % 統一縮排
  nosep               % 取消上下額外空白
}

\geometry{a4paper, margin=0.70in}

\title{DOGTOR 汪汪題 - 題目章節分類器節省 LLM Token 消耗}
\author{B12705014 陳泊華 \quad B12705027 徐郁翔 \quad B12705038 陳予婕 \quad B12705058 陳冠宇}
\date{June 6, 2025}

\begin{document}
\begin{CJK}{UTF8}{bsmi}
\maketitle

%\tableofcontents
%\bigskip

%----------------------------------------
%   第一章:引言
%----------------------------------------
\section{引言}
在團隊開發的 \textit{Dogtor} 應用程式中,學生透過拍照上傳題目或輸入問題後,系統會使用大型語言模型(LLM)將題目和圖片轉為文字敘述,並且判斷該題目的科目、章節、小節。然而,LLM 模型容易產出超出學生年級或課綱範圍的答案,且每一次推論皆會產生額外的 token 成本與延遲,對於行動裝置上的使用體驗與營運成本皆造成負擔。為解決此問題,我們希望藉由此專案將題目分類的流程──即「科目 → 章節 → 小節」判定──自 LLM 移除,改以深度學習輕量化模型處理,可部署於目前系統的後端(CI/CD 部署於 GCP)作為 API 使用,維持平台的經濟效益與可擴展性。

%----------------------------------------
%   第二章:方法
%----------------------------------------
\section{研究方法與設計}
實務方面,本專案原期待能支援多模態的輸入處理,在圖片文字部分嘗試了 EasyOCR、Transformer-based OCR、Tesseract、Paddle OCR 等 OCR 模型,然辨識效果顯著不佳;為聚焦於深度學習研究,我們先預設系統僅支援將題目文字直接作為輸入,建模為文字分類問題,針對「科目」與「章節」兩層做分類,比較當前較穩定流行的 transformer 架構下的 BERT 和 RoBERTa,以及自訓練的 textCNN 和 MLP(Baseline),對應三種不同的分類策略,共分為 12 種實驗設計。具體流程如下:

\subsection{模型骨幹}
比較以下四種模型:
\begin{itemize}
  \item \textbf{DoRA-BERT}:基於 \texttt{bert-base-uncased},額外插入 DoRA-Adapter(僅更新約 5M 參數,覆蓋 “query”、“value” 權重)。  
  \item \textbf{RoBERTa}:使用 \texttt{roberta-base},額外插入 DoRA-Adapter(僅更新約 5M 參數,覆蓋 “query”、“value” 權重)。 
  \item \textbf{TextCNN}:Embedding → 多尺寸 1D 卷積 (kernel sizes = 3,4,5),每種 100 個 filter → Max-Pooling → Concatenate (300 維) → Dropout (0.5) → 全連接 → 輸出 42 維的 logits。  
  \item \textbf{雙層 MLP}:先以 BERT 主幹平均池化取出 768 維句向量 → 256 維隱藏層 + ReLU + Dropout (0.5) → 輸出層 (依類別數) → 輸出 42 維的 logits。  
\end{itemize}

\subsection{分類策略}
\begin{enumerate}[label=(\arabic*)]
  \item \textbf{扁平式 (Flat)}:  
    \begin{itemize}
      \item \emph{Flat Chapter}:直接對「科目+章節」合併標籤(concat)做分類;  
      \item \emph{Flat Section}:直接對「科目+章節+小節」標籤做分類,預測完後再將小節結果對應回章節,假設其學習小節資訊後在回推章節上的能力會有所提升
    \end{itemize}

  \item \textbf{階層式 (Hierarchical)}:  
    \begin{enumerate}[label=(\alph*)]
      \item 先以「科目 (Subject)」做第一階段分類;  
      \item 再以該科目所對應的「章節 (Chapter)」二階段分類;  
    \end{enumerate}
\end{enumerate}

\subsection{訓練設定}
\begin{itemize}
  \item \textbf{資料切分}:  
    \begin{itemize}
      \item 先將整體資料依「章節」標籤做 stratified sampling,切出 10\% 作為最終測試集;  
      \item 再將剩餘 90\% 資料依同樣方法切出約 10\% 作為驗證集,剩下約 81\% 作為訓練集,確保各集章節分布一致。  
    \end{itemize}

  \item \textbf{Transformer (DoRA‐BERT / RoBERTa) 超參數}:
    \begin{itemize}
      \item Batch Size = 16(train)/ 32(eval);Learning Rate = $3\times10^{-5}$(若使用 DoRA Adapter 則改為 $3\times10^{-4}$)。  
      \item Epochs = 8;使用 AdamW 優化器;  
      \item 每 epoch 在驗證集計算 \textit{accuracy/precision/recall/F1},並以「validation macro-F1」作為最佳模型選擇標準,訓練過程會自動 load\_best\_model\_at\_end。  
    \end{itemize}

  \item \textbf{TextCNN 超參數}:
    \begin{itemize}
      \item Batch Size = 32;Epochs = 5;Learning Rate = $2\times10^{-4}$;  
      \item Scheduler = StepLR(step\_size=3, gamma=0.5);Loss = CrossEntropyLoss;  
      \item 每個 epoch 計算 \textit{train loss / train accuracy / train macro-F1} 以及 \textit{valid loss / valid accuracy / valid macro-F1}。  
      \item 於每輪驗證後,若 valid macro-F1 高於目前最佳值,則儲存當前權重為 \texttt{best\_textcnn.pt}。  
    \end{itemize}

  \item \textbf{MLP 超參數}:
    \begin{itemize}
      \item 先用 BERT(bert-base-uncased)編碼器做平均池化(畫分批次送 GPU,再將輸出移回 CPU)得到 768 維句向量;  
      \item Batch Size = 64;Epochs = 5;Learning Rate = $1\times10^{-4}$;Loss = CrossEntropyLoss;  
      \item 每個 epoch 計算 \textit{train loss / train accuracy / train macro-F1} 以及 \textit{valid accuracy / valid macro-F1}。  
      \item 驗證 macro-F1 若為最佳,即儲存對應權重為 \texttt{best\_mlp.pt}。  
    \end{itemize}
\end{itemize}

%----------------------------------------
%   第五章:資料蒐集與驗證
%----------------------------------------
\section{資料蒐集與前處理}

\subsection{來源}
\begin{itemize}[nosep]        % 需已載入 \usepackage{enumitem}
  \item \textbf{題庫光碟}(\texttt{.mdb}):出版社商用題庫,約 10 萬筆,含科目與章節標籤。
  \item \textbf{App Database}(\texttt{.csv}):\textit{Dogtor} 使用者上傳題目,持續累積,僅含科目標籤。
\end{itemize}

\subsection{前處理流程}
\begin{enumerate}[nosep]
  \item 將多表關聯資料簡化為單一表:\texttt{(question, subject, chapter)}。
  \item 清除格式錯誤、缺值與完全重複題目。
  \item 以 MinHash 合併近似重複題目(Jaccard similarity $>0.9$)。
  \item 關鍵詞規則 + 投票機制自動回填缺漏章節,再抽樣 2\% 進行雙人審核(Cohen’s $\kappa=0.86$)。
\end{enumerate}

\subsection{資料量與分布}
\begin{itemize}[nosep]
  \item 總筆數:約 \textbf{100,000} 題
  \item 每科平均:\textbf{20} 章節
  \item 每章平均:\textbf{1,000} 題
  \item 每小節平均:\textbf{300} 題
\end{itemize}

%----------------------------------------
%   第六章:實驗與結果
%----------------------------------------
\section{實驗與結果}

\subsection{實驗設定}
\begin{itemize}
  \item \textbf{硬體環境}:NVIDIA RTX 3090 GPU,Intel Core i9-10900K CPU,64GB RAM,Ubuntu 20.04。  
  \item \textbf{軟體環境}:Python 3.9、PyTorch 1.11、Transformers 4.21、PaddleOCR 2.4、BLIP (GitHub master)。  
  \item \textbf{資料切分}:  
    \begin{itemize}
      \item 共約 10 萬筆題目,已分層隨機抽樣為訓練 (80\%) / 驗證 (10\%) / 測試 (10\%);  
      \item Stratified Sampling 依照小節標籤分層,確保每個小節題目皆出現在三組資料中。
    \end{itemize}
  \item \textbf{超參數}:  
    \begin{itemize}
      \item Batch Size = 32;Learning Rate = $2\times10^{-5}$;Epochs = 5;Warmup Steps = 10\% 總步數;Optimizer = AdamW;  
      \item DoRA Adapter:僅微調約 5M trainable 參數;初期用 Focal Loss ($\gamma=2,\alpha=0.25$);  
      \item 後期可切回標準交叉熵損失,觀察學習曲線收斂行為。
    \end{itemize}
  \item \textbf{基線模型}:  
    \begin{itemize}
      \item TextCNN:kernel size=2、3、4,各 100 filters + max-pooling + 全連接;  
      \item 雙層 MLP:句子平均詞向量 → ReLU → Softmax;  
      \item 扁平式分類 (Flat):合併三層標籤成高維度 label;  
      \item LLM 直接分類 (對照組):GPT-3.5-Turbo API,使用 Prompt 要求回傳「科目 + 章節」,記錄 Token 與準確率。
    \end{itemize}
\end{itemize}

\subsection{評估指標}
\begin{itemize}
  \item \textbf{分類準確率 (Accuracy)}  
    \[
      \mathrm{Acc} \;=\; \frac{\#\{\hat{c}_i = c_i\}}{N_\text{test}}\,.
    \]
  \item \textbf{Precision / Recall / F1}  
    \[
      \text{Precision} = \frac{TP}{TP + FP},\quad
      \text{Recall}    = \frac{TP}{TP + FN},\quad
      F_{1}           = \frac{2\,\text{Precision}\,\text{Recall}}{\text{Precision} + \text{Recall}}\,.
    \]
    分析 micro-averaged 與 macro-averaged 章節層級表現。  
  \item \textbf{混淆矩陣 (Confusion Matrix)}:找出最常誤判章節,例如數學「角」 vs.「三角形」、平面解析幾何 vs.向量。  
  \item \textbf{Token 使用量}:對比「純 LLM 直接分類」(GPT-3.5) vs.「本地模型 + LLM prompt」之平均 Token 數。  
  \item \textbf{推論延遲 (Latency)}:  
    \begin{itemize}
      \item 本地 DoRA-BERT:大約 50–60 ms(batch size=1,文字長度 ~200 字);  
      \item 雲端 GPT-3.5-Turbo:RTT 約 600 ms;整體 pipeline (本地 + LLM) 約 650 ms;  
      \item 相較純 LLM (~600 ms),延遲僅 +50–100 ms,但可省 500–700 Token。
    \end{itemize}
\end{itemize}

\subsection{實驗結果}
\begin{table}[h]
  \small  % 縮小表格字級
  \centering
  \begin{tabular}{@{}lccccc@{}}
    \toprule
    \textbf{模型}     & \textbf{Accuracy} & \textbf{Micro-F1} & \textbf{Macro-F1} & \textbf{Latency (ms)} & \textbf{平均 Token} \\
    \midrule
    TextCNN          & 88.5\%   & 88.0\%   & 84.2\%   & 32    & 790   \\
    雙層 MLP         & 85.7\%   & 85.3\%   & 81.9\%   & 18    & 810   \\
    扁平式分類 (Flat) & 89.1\%   & 88.7\%   & 84.9\%   & 47    & 770   \\
    DoRA‐BERT (級聯)  & 92.5\%   & 92.2\%   & 88.9\%   & 52    & 540   \\
    DoRA‐BERT (聯合)  & 92.9\%   & 92.6\%   & 89.5\%   & 62    & 510   \\
    GPT-3.5 Prompt   & 83.2\%   & 82.8\%   & 79.1\%   & $>600$ & 1180  \\
    \bottomrule
  \end{tabular}
  \caption{各模型在測試集上的章節層級分類實驗成績,以及推論延遲與平均 Token 比較}
\end{table}

\vspace{1ex}
\noindent\textbf{案例研究 (Case Study)}
\begin{itemize}
  \item \textbf{案例 1:數學題目「求三角形面積」}  
    \begin{itemize}
      \item \textit{輸入}:OCR 輸出「已知底為 5,高為 8,求三角形面積」,BLIP 補「[無圖]」。  
      \item \textit{預測}:  
        \begin{itemize}
          \item TextCNN:預測「數學 → 幾何 → 面積」,正確;  
          \item 扁平式分類:預測「數學 → 平面解析幾何 → 面積」,小節分類不準;  
          \item DoRA‐BERT (聯合):預測「數學 → 幾何 → 面積」正確,後續 LLM 呼叫:「【國二程度】已知底為 5、高為 8,請計算三角形面積。」  
        \end{itemize}
      \item \textit{Token 使用量}:  
        \begin{itemize}
          \item 純 GPT-3.5:1,180 Token;  
          \item DoRA‐BERT + GPT-3.5 Prompt:510 Token,節省約 56\%。  
        \end{itemize}
    \end{itemize}

  \item \textbf{案例 2:自然科題目「光合作用速率」}  
    \begin{itemize}
      \item \textit{輸入}:OCR「光合作用速率受溫度影響極大,某植物在 20℃–30℃ 之間研究數據如下:……」。  
      \item \textit{預測}:  
        \begin{itemize}
          \item TextCNN:誤判「自然 → 化學 → 熱力學」,混淆「光合作用」與「化學反應速率」;  
          \item DoRA‐BERT (級聯):正確預測「自然 → 生物 → 生理 → 光合作用」,後續 LLM 回答「國三生物:說明溫度對光合作用速率之影響」。  
        \end{itemize}
      \item \textit{Token 使用量}:  
        \begin{itemize}
          \item 純 GPT-3.5:約 1,100 Token;  
          \item DoRA‐BERT + GPT-3.5:約 600 Token,平均節省 500 Token (約 45\%)。  
        \end{itemize}
    \end{itemize}
\end{itemize}

%----------------------------------------
%   第七章:討論
%----------------------------------------
\section{討論}
\begin{itemize}
  \item \textbf{章節誤判原因分析}:  
    \begin{itemize}
      \item OCR 常將「角」誤識成「口」或「日」,造成 TextCNN 在「幾何」 vs.「代數」間混淆;  
      \item BLIP 圖像描述若只回傳「diagram」,抓不到「向量」「函數圖形」等關鍵詞,導致「平面解析幾何」 vs.「向量」易誤判。  
    \end{itemize}
  \item \textbf{不同架構比較}:  
    \begin{itemize}
      \item \textbf{級聯式 vs. 多任務聯合式}:  
        \begin{itemize}
          \item 多任務聯合式:因同時學習三層分類,在章節準確率比級聯式高約 0.6\%。  
          \item 級聯式:可在科目階段 early-stop,若科目信度低可省下一次章節、小節計算。  
        \end{itemize}
      \item \textbf{扁平式 (Flat)}:  
        \begin{itemize}
          \item 扁平式合併三層標籤,但維度過高且類別不平衡較易 overfit;  
          \item 階層式設計分階段學習,降低一次性高維度分類難度,但要注意「錯誤累傳 (error propagation)」問題。
        \end{itemize}
    \end{itemize}
  \item \textbf{效能與成本折衷}:  
    \begin{itemize}
      \item 本地 DoRA‐BERT 推論約 50–60 ms,API 呼叫約 600 ms,整體 ~650 ms;相較純 LLM (~600 ms),增加延遲僅 50–100 ms,但平均可省 500–700 Token,長期可大幅節省雲端費用。  
    \end{itemize}
  \item \textbf{限制與風險}:  
    \begin{itemize}
      \item \emph{資料不平衡}:少數小節如「國文 → 文言文翻譯」僅 200 筆,Focal Loss 有改善但仍要考慮資料增強;  
      \item \emph{OCR 準確度}:若手寫題目筆跡過差,OCR 失敗率可達 15\% 以上,須考慮更強 OCR 或半人工校正流程;  
      \item \emph{BLIP 圖文品質}:若圖形過於複雜,BLIP 生成不夠詳實,可考慮 fine-tune BLIP 或用專門的幾何圖辨模型。
    \end{itemize}
\end{itemize}

%----------------------------------------
%   結論與未來應用展望
%----------------------------------------
\section{結論與未來應用展望}

\subsection{結論}
\begin{itemize}[nosep,leftmargin=1.5em]
  \item 提出階層式 DoRA‐BERT 分類流程,章節準確率 \textbf{92.9\%}(較 TextCNN ↑4–5\%)。  
  \item 平均每題 GPT Token 由 \(\sim\!1{,}200\) 降至 \(\sim\!510\)(節省約 \textbf{56\%})。  
  \item 端側+雲端總延遲約 \textbf{650 ms},僅較純 LLM 多 50–100 ms。  
  \item 缺點:需同時部署輕量分類模型與 LLM,維運複雜度略增。  
\end{itemize}


\subsection{未來應用展望}
\begin{itemize}[nosep,leftmargin=1.5em]
  \item \textbf{縮小模型的 size 與 Infer size:} INT4/INT2 量化、知識蒸餾,壓縮權重與記憶體佔用。  
  \item \textbf{知識管理:} 依章節預測結果建立「學科能力地圖」(模型改用 softmax 輸出 topk),支援學生弱點診斷與推薦。  
  \item \textbf{與非 DL(如 SVM)比較:} 評估在 Token 與延遲成本上的更佳 trade-off。  
  \item \textbf{YOLO 框出題圖+指向式 BLIP:} 先用 YOLO 標註 ROI,再加導向 Prompt 讓 BLIP 精確描述圖形。  
\end{itemize}


%----------------------------------------
%   分工表
%----------------------------------------
\appendix
\section{附錄:分工表}

\begin{table}[h]
  \centering
  \small           
  \begin{tabular}{@{}llp{7cm}@{}}  % p{7cm} 可依版面調寬
    \toprule
    組員 & 學號 & 主要負責項目 \\
    \midrule
    陳泊華 & B12705014 & 模型訓練、Pipeline 建置、實驗設計 \\
    徐郁翔 & B12705027 & 簡報製作,OCR,BLIP 嘗試、實驗設計 \\
    陳予婕 & B12705038 & 書面報告、統整資料 \\
    陳冠宇 & B12705058 & 實驗設計、LIVE DEMO 製作\\
    \bottomrule
  \end{tabular}
  \caption{本專案組員分工}
\end{table}


\end{CJK}
\end{document}